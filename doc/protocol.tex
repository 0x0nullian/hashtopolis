\documentclass{article}
\author{seinlc}
\usepackage[T1]{fontenc}
\usepackage[utf8]{inputenc}
\usepackage[left=10mm, right=10mm]{geometry}
\usepackage[english]{babel}
\begin{document}
	\title{Hashtopussy Communication Protocol}
	\maketitle
	\section*{Introduction}
	The communication between Hashtopussy Clients and Server is always in JSON formatted values. When sending a request to the server, it should be a POST with value
	"query=[JSON]".

	\section*{Errors}
	In case of an error with the query which the client sends to the server, the response will have following format with the corresponding action which was requested and the error message which should help in getting information about this error.
	\begin{verbatim}
	{
	  "action":"register",
	  "response":"ERROR",
	  "message":"Provided voucher does not exist."
	}
	\end{verbatim}
	\pagebreak
	\section*{Commands}
	
	\subsection*{Connection Check}
	To test the connection to the server or to test if an url is valid, you can issue a test action where the server just should respons with success.
	\begin{verbatim}
	{
	  "action":"test"
	}
	\end{verbatim}
	Response:
	\begin{verbatim}
	{
	  "action":"test",
	  "response":"SUCCESS"
	} 
	\end{verbatim}

	\subsection*{Register}
	When registering a new client to the server, following query should be used:
	(Important: To make sure an agent will be recognized as GPU agent, you have to get some info in the 'gpus' field. The server searches for keywords like 'ati', 'nvidia' etc.)
	\begin{verbatim}
	{
	  "action":"register",
	  "voucher":"89GD78tf",
	  "uid":"230-34-345-345",
	  "name":"client name",
	  "os":0,
	  "gpus":[
	    "ATI HD7970",
	    "ATI HD7970"
	  ]
	}
	\end{verbatim}
	All values are required and must not be empty. The operating system should be set as following: 0 => Linux, 1 => Windows, 2 => Mac OS X.\\
	As response the server will send following on success:
	\begin{verbatim}
	{
	  "action":"register",
	  "response":"SUCCESS",
	  "token":"GHhgdf(/&"
	} 
	\end{verbatim}
	The token will be the one, the client should use for his following connections on this server.
	
	\subsection*{Login}
	The client logs in to the server.
	\begin{verbatim}
	{
	  "action":"login",
	  "token":"GHhgdf(/&"
	}
	\end{verbatim}
	As response, the client will get back the agent timeout setting of the server (in seconds) if the login
	was successful.
	\begin{verbatim}
	{
	  "action":"login",
	  "response":"SUCCESS",
	  "timeout":30
	}
	\end{verbatim}
	
	\subsection*{Update}
	This is used by the client to check if there is an update available for the client script. Type specifies the type of the client, currently only 'csharp'.
	\begin{verbatim}
	{
	  "action":"update",
	  "version":"0.1.0 ALPHA",
	  "type":"csharp"
	}
	\end{verbatim}
	The server responds either with
	\begin{verbatim}
	{
	  "action":"update",
	  "response":"SUCCESS",
	  "version":"OK"
	}
	\end{verbatim}
	if the client has the newest version. Or he sends
	\begin{verbatim}
	{
	  "action":"update",
	  "response":"SUCCESS",
	  "version":"NEW",
	  "url":"https://example.com/getclient.php"
	}
	\end{verbatim}
	with 'data' containing the new version of the script.
	
	\subsection*{Download}
	This command is used to either download the 7z binary to extract Hashcat, or to get information where to download the newest Hashcat version and some additional informations about it.
	\begin{verbatim}
	{
	  "action":"download",
	  "type":"7zr",
	  "token":"GHhgdf(/&"
	}
	\end{verbatim}
	The server then will send a base64 encoded binary (attribute 'executable') encapsulated in JSON:
	\begin{verbatim}
	{
	  "action":"download",
	  "response":"SUCCESS",
	  "executable":"a2prbGhqa2doa.....2xrZTNnNDM0Zw=="
	}
	\end{verbatim}
	To check if a new Hashcat binary is available, following command is used.
	\begin{verbatim}
	{
	  "action":"download",
	  "type":"hashcat",
	  "token":"GHhgdf(/&",
	  "force":1
	}
	\end{verbatim}
	'force' is optional and is be used when the server should send version information anyway, even if the client already has the newest version. 'version' contains the current version the client has.\\
	The server will respond with following information:
	\begin{verbatim}
	{
	  "action":"download",
	  "response":"SUCCESS",
	  "version":"NEW",
	  "url":"https://hashcat.net/files/hashcat-3.01.7z",
	  "files":[
	    "hashcat.hctune",
	    "hashcat64.bin"
	  ],
	  "rootdir":"hashcat-3.0.1",
	  "executable":"hashcat64.bin"
	}
	\end{verbatim}
	
	\subsection*{Error}
	In case there happens an error with Hashcat on the client, he can submit the error message to the server where it will be assigned to the agent and shown on the page.
	\begin{verbatim}
	{
	  "action":"error",
	  "task":45,
	  "token":"GHhgdf(/&",
	  "message":"This is a sample error message!"
	}
	\end{verbatim}
	The server will confirm the error messages:
	\begin{verbatim}
	{
	  "action":"error",
	  "response":"SUCCESS"
	}
	\end{verbatim}
	
	\subsection*{Get File}
	If the client needs a file for running a task (Wordlist or Rule File) he needs to request the url for downloading it.
	\begin{verbatim}
	{
	  "action":"file",
	  "token":"GHhgdf(/&",
	  "task":56,
	  "file":"rockyou.txt"
	}
	\end{verbatim}
	The server sends then the url, where the client can download the file from.
	\begin{verbatim}
	{
	  "action":"file",
	  "response":"SUCCESS",
	  "url":"get.php?file=5&token=GHhgdf(/&"
	}
	\end{verbatim}
	
	\subsection*{Get Hashlist}
	The client can download a Hashlist/Superhashlist.
	\begin{verbatim}
	{
	  "action":"hashes",
	  "token":"GHhgdf(/&",
	  "hashlist":23
	}
	\end{verbatim}
	If there occurs any error with this request, the server will respond in JSON format with error information. If everything is correct, it will send a list of hashes which were requested.\\\\
	The server then will send the whole hashlist as plain-text download on success, otherwise he will send an error.
	
	\subsection*{Get Task}
	The client requests the current task he should work on.
	\begin{verbatim}
	{
	  "action":"task",
	  "token":"GHhgdf(/&"
	}
	\end{verbatim}
	The server will send then a task to do (benchType can be 'speed' or 'run'):
	\begin{verbatim}
	{
	  "action":"task",
	  "response":"SUCCESS",
	  "task": 25,
	  "attackcmd": "#HL# -a 3 ?a?a?a?a",
	  "cmdpars": "--hash-type=0",
	  "hashlist": 13,
	  "bench": 30,
	  "statustimer": 5,
	  "benchType":"speed",
	  "files": [
	    "rockyou.txt",
	    "wordlist.txt"
	  ]
	}
	\end{verbatim}
	or send this if there is no task available:
	\begin{verbatim}
	{
	  "action":"task",
	  "response":"SUCCESS",
	  "task":"NONE"
	}
	\end{verbatim}
	
	\subsection*{Get Chunk}
	The client requests the current task he should work on.
	\begin{verbatim}
	{
	  "action":"chunk", 
	  "token":"GHhgdf(/&",
	  "taskId": 5
	}
	\end{verbatim}
	The server will send then a chunk to work on:
	\begin{verbatim}
	{
	  "action":"chunk",
	  "response":"SUCCESS",
	  "status":"OK",
	  "chunk": 13,
	  "skip": 45000,
	  "length": 10000
	}
	\end{verbatim}
	or he can send one of the following answers which should be done then:
	
	In case it is required to measure the keyspace of the corresponding task.
	\begin{verbatim}
	{
	  "action":"chunk",
	  "response":"SUCCESS",
	  "status":"keyspace_required"
	}
	\end{verbatim}
	
	In case the current task is already fully dispatched and the agent should request for a new task.
	\begin{verbatim}
	{
	  "action":"chunk",
	  "response":"SUCCESS",
	  "status":"fully_dispatched"
	}
	\end{verbatim}
	
	In case there is no benchmark for this task on the current agent. So it should start benchmarking then.
	\begin{verbatim}
	{
	  "action":"chunk",
	  "response":"SUCCESS",
	  "status":"benchmark"
	}
	\end{verbatim}
	
	
	\subsection*{Send Keyspace}
	The client sends the calculated keyspace of a task.
	\begin{verbatim}
	{
	  "action":"keyspace",
	  "token":"GHhgdf(/&",
	  "taskId":35,
	  "keyspace":568000
	}
	\end{verbatim}
	The server should reply following on success:
	\begin{verbatim}
	{
	  "action":"keyspace",
	  "response":"SUCCESS",
	  "keyspace": "OK"
	}
	\end{verbatim}
	
	\subsection*{Send Benchmark}
	The client sends the tested benchmark of a task. Benchmark type can be 'speed' or 'run'. If the 'speed' benchmark type is used, there are two values to be sent separated by ':'. \\
	The 'speed' benchmark type is using the --progress-only switch of hashcat. \\
	The 'run' benchmark type uses the old method of hashtopussy to benchmark a task in running the given task for some time.
	\begin{verbatim}
	{
	  "action":"bench",
	  "token":"GHhgdf(/&",
	  "taskId":35,
	  "type": "run",
	  "result": "674"
	}
	\end{verbatim}
	\begin{verbatim}
	{
	  "action":"bench",
	  "token":"GHhgdf(/&",
	  "taskId":35,
	  "type": "speed",
	  "result": "674:674.74"
	}
	\end{verbatim}
	The server should reply following on success:
	\begin{verbatim}
	{
	  "action":"bench",
	  "response":"SUCCESS",
	  "benchmark": "OK"
	}
	\end{verbatim}
	
	
	\subsection*{Send Solve}
	In the given interval of the task, the client is working on, it should report the progress in following way:
	\begin{verbatim}
	{
	  "action":"solve",
	  "token":"GHhgdf(/&",
	  "chunk":35,
	  "keyspaceProgress":568000,
	  "progress":500,
	  "total":600,
	  "speed":570000,
	  "state": 3,
	  "cracks": [
	    "7fde65673fd28736423f23423786f:thisisplain",
	    "7fde65673f28987f7423f2342378f:otherplain",
	  ]
	}
	\end{verbatim}
	The server will reply with following:
	\begin{verbatim}
	{
	  "action":"solve",
	  "response":"SUCCESS",
	  "cracked": 2,
	  "skipped": 0
	}
	\end{verbatim}
	If there were cracks on the same hashlist from some other clients or view the web-interface, the server will send the hashes which should be zapped from the hashlist
	\begin{verbatim}
	{
	  "action":"solve",
	  "response":"SUCCESS",
	  "cracked": 2,
	  "skipped": 0,
	  "zap": [
	    "87634876fe7f7df5e76f56a757",
	    "7863487563874ffefe67575745"
	  ]
	}
	\end{verbatim}
\end{document}