\documentclass{article}
\author{seinlc}
\usepackage[T1]{fontenc}
\usepackage[utf8]{inputenc}
\usepackage[english]{babel}
\begin{document}
	\title{Hashtopussy Communication Protocol}
	\maketitle
	\section*{Introduction}
	The communication between Hashtopussy Clients and Server is always in JSON formatted values. When sending a request to the server, it should be a POST with value
	"query=[JSON]".
	
	\section*{Errors}
	In case of an error with the query which the client sends to the server, the response will have following format with the corresponding action which was requested and the error message which should help in getting information about this error.
	\begin{verbatim}
	{
	  "action":"register",
	  "response":"ERROR",
	  "message":"Provided voucher does not exist."
	}
	\end{verbatim}
	\pagebreak
	\section*{Commands}
	
	\subsection*{Register}
	When registering a new client to the server, following query should be used:
	\begin{verbatim}
	{  
	  "action":"register",
	  "voucher":"89GD78tf",
	  "uid":"230-34-345-345",
	  "name":"client name",
	  "os":0,
	  "gpus":[  
	    "ATI HD7970",
	    "ATI HD7970"
	  ]
	}
	\end{verbatim}
	All values are required and must not be empty. The operating system should be set as following: 0 => Linux, 1 => Windows, 2 => Mac OS X.
	
	As response the server will send following on success:
	\begin{verbatim}
	{
	  "action":"register",
	  "response":"SUCCESS",
	  "token":"GHhgdf(/&"
	} 
	\end{verbatim}
	The token will be the one, the client should use for his following connections on this server.
	
	\subsection*{Login}
	The client logs in to the server.
	\begin{verbatim}
	{
	  "action":"login",
	  "token":"GHhgdf(/&"
	}
	\end{verbatim}
	As response, the client will get back the agent timeout setting of the server (in seconds) if the login
	was successful.
	\begin{verbatim}
	{
	  "action":"login",
	  "response":"SUCCESS",
	  "timeout":30
	}
	\end{verbatim}
	
	\subsection*{Update}
	This is used by the client to check if there is an update available for the client script.
	\begin{verbatim}
	{
	  "action":"update",
	  "version":"0.1.0 ALPHA"
	}
	\end{verbatim}
	The server responds either with
	\begin{verbatim}
	{
	  "action":"update",
	  "response":"SUCCESS",
	  "version":"OK"
	}
	\end{verbatim}
	if the client has the newest version. Or he sends
	\begin{verbatim}
	{
	  "action":"update",
	  "response":"SUCCESS",
	  "version":"NEW",
	  "data":"........"
	}
	\end{verbatim}
	with 'data' containing the new version of the script.
	
	\subsection*{Download}
	This command is used to either download the 7z binary to extract Hashcat, or to get information where to download the newest Hashcat version and some additional informations about it.
	\begin{verbatim}
	{
	  "action":"download",
	  "type":"7zr",
	  "token":"GHhgdf(/&"
	}
	\end{verbatim}
	This will result in a non-JSON response. The server will simply provide the corresponding 7z binary for the agent as download.\\
	To check if a new Hashcat binary is available, following command is used.
	\begin{verbatim}
	{
	  "action":"download",
	  "type":"hashcat",
	  "token":"GHhgdf(/&",
	  "version":"3.0.1",
	  "force":1
	}
	\end{verbatim}
	'force' is optional and is be used when the server should send version information anyway, even if the client already has the newest version. 'version' contains the current version the client has.\\
	The server will respond with following information:
	\begin{verbatim}
	{
	  "action":"download",
	  "response":"SUCCESS",
	  "version":"NEW",
	  "url":"https://hashcat.net/files/hashcat-3.01.7z",
	  "files":[
	    "hashcat.hctune",
	    "hashcat64.bin"
	  ],
	  "rootdir":"hashcat-3.0.1",
	  "executable":"hashcat64.bin"
	}
	\end{verbatim}
\end{document}